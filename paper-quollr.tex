% !TeX root = RJwrapper.tex
\title{quollr: An R Package for Visualizing 2-D Models from Non-linear Dimension Reductions in High Dimensional Space}


\author{by Jayani P. Gamage, Dianne Cook, Paul Harrison, Michael Lydeamore, and Thiyanga S. Talagala}

\maketitle

\abstract{%
Non-linear dimension reduction (NLDR) methods provide a low-dimensional representation of high-dimensional data (\(p\text{-}D\)) by applying a non-linear transformation. However, the complexity of the transformations and data structures can create wildly different representations depending on the method and (hyper-)parameter choices. It is difficult to determine whether any of these representations are accurate, which one is the best, or whether they have missed important structures. The R package \CRANpkg{quollr} has been developed as a new visual tool to determine which method and which (hyper-)parameter choices provide the most accurate representation of high-dimensional data. The \texttt{scurve} data from the package is used to illustrate the algorithm. Single-cell RNA sequencing (scRNA-seq) data from mouse limb muscles are used to demonstrate the usability of the package.
}

\section{Introduction}\label{introduction}

Non-linear dimension reduction (NLDR) techniques, such as t-distributed stochastic neighbor embedding (tSNE) \citep{laurens2008}, uniform manifold approximation and projection (UMAP) \citep{leland2018}, potential of heat-diffusion for affinity-based trajectory embedding (PHATE) algorithm \citep{moon2019}, large-scale dimensionality reduction Using triplets (TriMAP) \citep{amid2019}, and pairwise controlled manifold approximation (PaCMAP) \citep{yingfan2021}, create wildly different representations depending on the selected method and (hyper-)parameter choices. It is difficult to determine whether any of these representations are accurate, which one is the best, or whether they have missed important structures.

This paper presents the R package, \texttt{quollr}, which is useful for understanding how NLDR warps high-dimensional space and fits the data.

\includegraphics[width=1\linewidth]{paper-quollr_files/figure-latex/unnamed-chunk-3-1}

The paper is organized as follows. The next section introduces the implementation of the \texttt{quollr} package on CRAN, including a demonstration of the package's key functions and visualization capabilities. In the application section, we illustrate the algorithm's functionality for studying a clustering data structure. Finally, we conclude the paper with a brief summary and discuss potential opportunities for using our algorithm.

\bibliography{paper-quollr.bib}

\address{%
Jayani P. Gamage\\
Monash University\\%
Department of Econometrics and Business Statistics, VIC 3800 Australia\\
%
\url{https://jayanilakshika.netlify.app/}\\%
\textit{ORCiD: \href{https://orcid.org/0000-0002-6265-6481}{0000-0002-6265-6481}}\\%
\email{jayani.piyadigamage@monash.edu}%
}

\address{%
Dianne Cook\\
Monash University\\%
Department of Econometrics and Business Statistics, VIC 3800 Australia\\
%
\url{http://www.dicook.org/}\\%
\textit{ORCiD: \href{https://orcid.org/0000-0002-3813-7155}{0000-0002-3813-7155}}\\%
\href{mailto:dicook@monash.edu}{\nolinkurl{dicook@monash.edu}}%
}

\address{%
Paul Harrison\\
Monash University\\%
MGBP, BDInstitute, VIC 3800 Australia\\
%
%
\textit{ORCiD: \href{https://orcid.org/0000-0002-3980-268X}{0000-0002-3980-268X}}\\%
\href{mailto:paul.harrison@monash.edu}{\nolinkurl{paul.harrison@monash.edu}}%
}

\address{%
Michael Lydeamore\\
Monash University\\%
Department of Econometrics and Business Statistics, VIC 3800 Australia\\
%
%
\textit{ORCiD: \href{https://orcid.org/0000-0001-6515-827X}{0000-0001-6515-827X}}\\%
\href{mailto:michael.lydeamore@monash.edu}{\nolinkurl{michael.lydeamore@monash.edu}}%
}

\address{%
Thiyanga S. Talagala\\
University of Sri Jayewardenepura\\%
Department of Statistics, Gangodawila, Nugegoda 10100 Sri Lanka\\
%
\url{https://thiyanga.netlify.app/}\\%
\textit{ORCiD: \href{https://orcid.org/0000-0002-0656-9789}{0000-0002-0656-9789}}\\%
\href{mailto:ttalagala@sjp.ac.lk}{\nolinkurl{ttalagala@sjp.ac.lk}}%
}
