% !TeX root = RJwrapper.tex
\title{quollr: An R Package for Visalizing 2D Models in High Dimensional Space}


\author{by Jayani P.G. Lakshika, Dianne Cook, Paul Harrison, Michael Lydeamore, and Thiyanga S. Talagala}

\maketitle

\abstract{%
An abstract of less than 150 words.
}

\begin{verbatim}
library(quollr)
library(tibble)
library(knitr)
library(kableExtra)
library(ggplot2)

set.seed(20230531)
\end{verbatim}

\hypertarget{introduction}{%
\section{Introduction}\label{introduction}}

\hypertarget{methodology}{%
\section{Methodology}\label{methodology}}

\hypertarget{usage}{%
\subsection{Usage}\label{usage}}

\begin{itemize}
\item
  dependencies
\item
  basic example
\end{itemize}

\hypertarget{datasets}{%
\subsubsection{Datasets}\label{datasets}}

The \texttt{quollr} package comes with several data sets that load with the package. These are described in Table \ref{tab:datasets-tb-pdf}.

\begin{table}

\caption{\label{tab:datasets-tb-pdf}quollr datasets}
\centering
\begin{tabular}[t]{>{\raggedright\arraybackslash}p{4cm}>{\raggedright\arraybackslash}p{8cm}}
\toprule
data & explanation\\
\midrule
s\_curve\_noise & Simulated 3D S-curve data with additional four noise dimensions.\\
s\_curve\_noise\_training & Training data derived from S-curve data.\\
s\_curve\_noise\_test & Test data derived from S-curve data.\\
s\_curve\_noise\_umap & UMAP 2D embedding data of S-curve data (n\_neighbors: 15, min\_dist: 0.1).\\
s\_curve\_noise\_umap\_predict & Predicted UMAP 2D embedding data of S-curve data\\
\addlinespace
s\_curve\_noise\_umap\_scaled & Scaled UMAP 2D embedding data of S-curve data\\
\bottomrule
\end{tabular}
\end{table}

\hypertarget{preprocessing}{%
\subsection{Preprocessing}\label{preprocessing}}

In the preprocessing stage, we aim to prepare our data to fit within the bounds required for regular hexagonal binning, ensuring effective visualization. To achieve this, we implement two key scaling steps. Firstly, we scale the first 2D embedding component to range between \(0\) and \(1\), ensuring that all data points fall within this normalized interval. Secondly, we scale the second 2D embedding component to range between \(0\) and \(y_{max}\). This adjustment helps maintain the necessary symmetry and spacing required for regular hexagons.

\begin{equation}\label{eq:aspect_ratio}
ar = \frac{r_2}{r_1} \tag{1}
\end{equation}

\begin{equation}\label{eq:y_max}
y_{max} = ceiling(\frac{ar}{hr}) * hr \tag{2}
\end{equation}

\begin{verbatim}
scaled_data <- gen_scaled_data(data = s_curve_noise_umap, x = "UMAP1", 
                y = "UMAP2", hex_ratio = NA)
glimpse(scaled_data)
\end{verbatim}

\begin{verbatim}
#> List of 2
#>  $ scaled_UMAP1: num [1:75] 0.0804 0.7386 0.8399 0.1672 0.2629 ...
#>  $ scaled_UMAP2: num [1:75] 0.366 1.1464 1.2392 0.0494 0.4556 ...
\end{verbatim}

\hypertarget{construct-the-2d-model}{%
\subsection{Construct the 2D model}\label{construct-the-2d-model}}

\hypertarget{compute-hexagonal-bin-configurations}{%
\subsubsection{Compute hexagonal bin configurations}\label{compute-hexagonal-bin-configurations}}

\hypertarget{construct-the-high-d-model}{%
\subsection{Construct the high-D model}\label{construct-the-high-d-model}}

\hypertarget{model-function}{%
\subsection{Model function}\label{model-function}}

The \texttt{fit\_highd\_model()} function is used to generate both the 2D and high-D models.

\begin{verbatim}
fit_highd_model(training_data = s_curve_noise_training, x = "UMAP1", y = "UMAP2",
nldr_df_with_id = s_curve_noise_umap_scaled, col_start_2d = "UMAP", col_start_highd = "x")
\end{verbatim}

\begin{verbatim}
#> $df_bin
#> # A tibble: 10 x 8
#>    hb_id       x1     x2     x3        x4        x5       x6        x7
#>    <dbl>    <dbl>  <dbl>  <dbl>     <dbl>     <dbl>    <dbl>     <dbl>
#>  1     2 -0.637   1.74   -1.76   0.00953  -0.00143  -0.0117  -0.00152 
#>  2     6 -0.498   0.524  -1.73  -0.000237  0.00234  -0.0297  -0.000242
#>  3     7  0.294   1.40   -1.88   0.00890  -0.00803  -0.0123  -0.00120 
#>  4    12  0.309   0.0421 -1.83   0.00656   0.00823   0.00489 -0.00389 
#>  5    13  0.868   0.747  -0.781 -0.00408   0.000857  0.0248   0.00170 
#>  6    18  0.357   1.27   -0.169  0.00607   0.00124   0.0152   0.00204 
#>  7    24 -0.792   1.25    0.514 -0.000777  0.000464  0.00602  0.000371
#>  8    28 -0.597   1.19    1.77   0.000240 -0.00417  -0.0185  -0.000786
#>  9    29 -0.00544 0.211   1.92   0.00116   0.00266   0.00949 -0.00209 
#> 10    34  0.622   1.21    1.64  -0.000560  0.00540  -0.00741 -0.000886
#> 
#> $df_bin_centroids
#>    hexID       c_x   c_y std_counts
#> 1      2 0.1732051 -0.15  0.2352941
#> 2      6 0.0000000  0.15  0.5294118
#> 3      7 0.3464102  0.15  0.4117647
#> 4     12 0.1732051  0.45  0.1764706
#> 5     13 0.5196152  0.45  0.3529412
#> 6     18 0.6928203  0.75  0.7058824
#> 7     24 0.8660254  1.05  0.4705882
#> 8     28 0.6928203  1.35  0.2941176
#> 9     29 1.0392305  1.35  0.2352941
#> 10    34 0.8660254  1.65  1.0000000
\end{verbatim}

\hypertarget{model-summaries}{%
\subsection{Model summaries}\label{model-summaries}}

\hypertarget{predict-2d-embeddings}{%
\subsubsection{Predict 2D embeddings}\label{predict-2d-embeddings}}

There are some of NLDR techniques that don't give any functions for predictions. In that sense our methodology facilitates to compute 2D embedding with our workflow.

\hypertarget{goodness-of-fit-statistics}{%
\subsubsection{Goodness of fit statistics}\label{goodness-of-fit-statistics}}

There are two Goodness of fit statistics were produced. MSE and AIC which interpret the model accuracy.

\hypertarget{visualizations}{%
\subsection{Visualizations}\label{visualizations}}

We use static visualizations to understand the model constructed in 2D. On the other hand, dynamic visualization is used to see how the model looks in high-D space.

\hypertarget{static-visualizations}{%
\paragraph{Static visualizations}\label{static-visualizations}}

Static visualizations main involves two types of results. One is the triangulation and the other is the long edge removal. Both types of visualizations provide ggplot objects.

\begin{itemize}
\item
  Triangulation result: To visualize the results of triangulation, we input a dataset containing hexagonal bin centroid coordinates where 2D embedding data exists. \texttt{geom\_trimesh()} is used to visualize this result.
\item
  Long edge removal: The long edge removal process involves identifying and removing long edges from the triangular mesh. Table shows the main arguments of the functions. We offer two functions for visualizing this process:
\item
  \texttt{colour\_long\_edges()}: This function colors the long edges within the triangular mesh by red.
\item
  \texttt{remove\_long\_edges()}: After identifying long edges, this function draws the triangular mesh without the long edges.
\end{itemize}

\ref{tab:lgvis-tb-pdf}

\begin{table}

\caption{\label{tab:lgvis-tb-pdf}The main arguments for `vis\_lg\_mesh()` and `vis\_rmlg\_mesh()`}
\centering
\begin{tabular}[t]{>{\raggedright\arraybackslash}p{4cm}>{\raggedright\arraybackslash}p{8cm}}
\toprule
argument & explanation\\
\midrule
.data & The data frame containing the edge information.\\
benchmark\_value & The threshold value to determine long edges.\\
triangular\_object & The triangular object containing the mesh information.\\
distance\_col & The column name in `.data` representing the distances.\\
\bottomrule
\end{tabular}
\end{table}

\hypertarget{dynamic-visaulizations}{%
\paragraph{Dynamic visaulizations}\label{dynamic-visaulizations}}

The \texttt{show\_langevitour()} function enables dynamic visualization of the 2D model alongside the high-dimensional (high-D) data in its original space. This visualization is facilitated by langevitour object, allowing users to interactively explore the relationship between the 2D embeddings and the underlying high-dimensional data. The main arguments are shown in Table \ref{tab:dyvis-tb-pdf}.

\begin{table}

\caption{\label{tab:dyvis-tb-pdf}The main arguments for `show\_langevitour()`}
\centering
\begin{tabular}[t]{>{\raggedright\arraybackslash}p{4cm}>{\raggedright\arraybackslash}p{8cm}}
\toprule
argument & explanation\\
\midrule
df & A data frame containing the high-dimensional data.\\
df\_b & A data frame containing the high-dimensional coordinates of bin centroids/means.\\
df\_b\_with\_center\_data & The dataset with hexbin centroids/ means.\\
benchmark\_value & The benchmark value used to remove long edges (optional).\\
distance\_df & The distance dataframe.\\
\addlinespace
distance\_col & The name of the distance column.\\
use\_default\_benchmark\_val & Logical, indicating whether to use default benchmark value  to remove long edges(default is FALSE).\\
column\_start\_text & The text that begin the column name of the high-dimensional data.\\
\bottomrule
\end{tabular}
\end{table}

\hypertarget{distance-with-bin-means}{%
\subsection{Distance with bin means}\label{distance-with-bin-means}}

\begin{verbatim}
bin_list <- calc_bins(data = s_curve_noise_umap_scaled, 
                      x = "UMAP1", y = "UMAP2", 
                      hex_size = NA, buffer_x = NA, 
                      buffer_y = NA)
num_bins_x <- bin_list$num_x
num_bins_y <- bin_list$num_y

hb_obj <- hex_binning(data = s_curve_noise_umap_scaled, 
                      x = "UMAP1", y = "UMAP2", 
                      num_bins_x = num_bins_x, num_bins_y = num_bins_y, 
                      x_start = NA, y_start = NA, 
                      buffer_x = NA, buffer_y = NA, 
                      hex_size = NA, col_start = "UMAP")


all_centroids_df <- as.data.frame(do.call(cbind, hb_obj$centroids))
counts_df <- as.data.frame(do.call(cbind, hb_obj$std_cts))
nldr_df_with_hex_id <- as.data.frame(do.call(cbind, hb_obj$data_hb_id))

## To obtain bin centroids
df_bin_centroids <- extract_hexbin_mean(nldr_df_with_hex_id = nldr_df_with_hex_id,
                                             counts_df = counts_df)

df_all <- dplyr::bind_cols(s_curve_noise_training |> dplyr::select(-ID), nldr_df_with_hex_id)

ggplot() + 
  geom_trimesh(data = df_bin_centroids, mapping = aes(x = c_x, y = c_y)) +
  coord_fixed()
\end{verbatim}

\includegraphics{paper-quollr_files/figure-latex/unnamed-chunk-9-1.pdf}

\begin{verbatim}
tr1_object <- tri_bin_centroids(hex_df = df_bin_centroids, x = "c_x", y = "c_y")
tr_from_to_df <- gen_edges(tri_object = tr1_object)
distance_df <- cal_2d_dist(tr_coord_df = tr_from_to_df, 
                           start_x = "x_from", start_y = "y_from", 
                           end_x = "x_to", end_y = "y_to", 
                           select_vars = c("from", "to", "distance"))


## averaged high-D data
df_bin <- weighted_highD_data(training_data = s_curve_noise_training, nldr_df_with_hex_id = nldr_df_with_hex_id, column_start_text = "x")

vis_lg_mesh(distance_edges = distance_df, benchmark_value = 1,
tr_coord_df = tr_from_to_df, distance_col = "distance")
\end{verbatim}

\includegraphics{paper-quollr_files/figure-latex/unnamed-chunk-9-2.pdf}

\begin{verbatim}
vis_rmlg_mesh(distance_edges = distance_df, benchmark_value = 1,
tr_coord_df = tr_from_to_df, distance_col = "distance")
\end{verbatim}

\includegraphics{paper-quollr_files/figure-latex/unnamed-chunk-9-3.pdf}

\begin{verbatim}
show_langevitour(df = df_all, df_b = df_bin, 
                 df_b_with_center_data = df_bin_centroids, 
                 benchmark_value = 1, distance = distance_df, 
                 distance_col = "distance", 
                 use_default_benchmark_val = FALSE, col_start = "x")
\end{verbatim}

\includegraphics{paper-quollr_files/figure-latex/unnamed-chunk-9-4.pdf}

\hypertarget{tests}{%
\subsection{Tests}\label{tests}}

All functions have tests written and implemented using the \CRANpkg{testthat} (Wickham 2011) in R.

\hypertarget{application}{%
\section{Application}\label{application}}

\hypertarget{conclusion}{%
\section{Conclusion}\label{conclusion}}

\hypertarget{acknowledgements}{%
\section{Acknowledgements}\label{acknowledgements}}

This article is created using \CRANpkg{knitr} (Xie 2015) and \CRANpkg{rmarkdown} (Xie, Allaire, and Grolemund 2018) in R with the \texttt{rjtools::rjournal\_article} template. The source code for reproducing this paper can be found at: \url{https://github.com/JayaniLakshika/paper-quollr}.

\hypertarget{references}{%
\section*{References}\label{references}}
\addcontentsline{toc}{section}{References}

\hypertarget{refs}{}
\begin{CSLReferences}{1}{0}
\leavevmode\vadjust pre{\hypertarget{ref-testthat}{}}%
Wickham, Hadley. 2011. {``Testthat: Get Started with Testing.''} \emph{The R Journal} 3: 5--10. \url{https://journal.r-project.org/archive/2011-1/RJournal_2011-1_Wickham.pdf}.

\leavevmode\vadjust pre{\hypertarget{ref-knitr}{}}%
Xie, Yihui. 2015. \emph{Dynamic Documents with {R} and Knitr}. 2nd ed. Boca Raton, Florida: Chapman; Hall/CRC. \url{https://yihui.name/knitr/}.

\leavevmode\vadjust pre{\hypertarget{ref-rmarkdown}{}}%
Xie, Yihui, J. J. Allaire, and Garrett Grolemund. 2018. \emph{{R} Markdown: The Definitive Guide}. Boca Raton, Florida: Chapman; Hall/CRC. \url{https://bookdown.org/yihui/rmarkdown}.

\end{CSLReferences}


\address{%
Jayani P.G. Lakshika\\
Monash University\\%
Department of Econometrics and Business Statistics, VIC 3800 Australia\\
%
\url{https://jayanilakshika.netlify.app/}\\%
\textit{ORCiD: \href{https://orcid.org/0000-0002-6265-6481}{0000-0002-6265-6481}}\\%
\email{jayani.piyadigamage@monash.edu}%
}

\address{%
Dianne Cook\\
Monash University\\%
Department of Econometrics and Business Statistics, VIC 3800 Australia\\
%
\url{http://www.dicook.org/}\\%
\textit{ORCiD: \href{https://orcid.org/0000-0002-3813-7155}{0000-0002-3813-7155}}\\%
\href{mailto:dicook@monash.edu}{\nolinkurl{dicook@monash.edu}}%
}

\address{%
Paul Harrison\\
Monash University\\%
MGBP, BDInstitute, VIC 3800 Australia\\
%
%
\textit{ORCiD: \href{https://orcid.org/0000-0002-3980-268X}{0000-0002-3980-268X}}\\%
\href{mailto:paul.harrison@monash.edu}{\nolinkurl{paul.harrison@monash.edu}}%
}

\address{%
Michael Lydeamore\\
Monash University\\%
Department of Econometrics and Business Statistics, VIC 3800 Australia\\
%
%
\textit{ORCiD: \href{https://orcid.org/0000-0001-6515-827X}{0000-0001-6515-827X}}\\%
\href{mailto:michael.lydeamore@monash.edu}{\nolinkurl{michael.lydeamore@monash.edu}}%
}

\address{%
Thiyanga S. Talagala\\
University of Sri Jayewardenepura\\%
Department of Statistics, Gangodawila, Nugegoda 10100 Sri Lanka\\
%
\url{https://thiyanga.netlify.app/}\\%
\textit{ORCiD: \href{https://orcid.org/0000-0002-0656-9789}{0000-0002-0656-9789}}\\%
\href{mailto:ttalagala@sjp.ac.lk}{\nolinkurl{ttalagala@sjp.ac.lk}}%
}
